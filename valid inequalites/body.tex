\section{Valid Inequality 1}
\begin{equation}
    \sum_{v \in V} x_{vi} \geq \left\lceil \frac{|V|}{k} \right\rceil
\end{equation}
\paragraph{Explanation} This constraint requires the first block of the partition to contain at least 
$\lceil |V|/K \rceil$ vertices. Since the vertex set $V$ is divided into exactly 
$K$ nonempty blocks, the average block size is $|V|/K$. By the pigeonhole principle, 
at least one block must contain at least $\lceil |V|/K \rceil$ vertices. Assigning 
this requirement to block $1$ is valid because of the symmetry-breaking constraints 
that enforce a nonincreasing block size order:
\[
\sum_{v \in V} x_{v,i} \;\geq\; \sum_{v \in V} x_{v,i+1}, \quad 
\forall i \in \{1,\dots,K-1\}.
\]
 
Thus, block $1$ can be designated as the largest block without loss of generality. 
This inequality preserves feasibility, strengthens the LP relaxation by ruling out 
artificially balanced fractional solutions, and reduces solver symmetry by eliminating 
weakly balanced partitions.
% -----------------------------------------------------------------

\section{Valid Inequality 2}
\begin{equation}
    \sum_{v \in V} x_{vi} \leq \left\lfloor \frac{|V| - K + i}{i} \right\rfloor \quad \forall i \in \Pi
\end{equation}
\paragraph{Explanation} Let $s_j := \sum_{v\in V} x_{v,j}$ denote the size of block $j$. The model enforces
(i) nonemptiness $s_j \ge 1$ for all $j$, (ii) a nonincreasing order
$s_1 \ge s_2 \ge \cdots \ge s_K$, and (iii) $\sum_{j=1}^K s_j = |V|$.
Fix $i$. Since the last $K-i$ blocks must contain at least $K-i$ vertices in total,
the first $i$ blocks can use at most $|V|-(K-i)$ vertices:
\[
\sum_{j=1}^{i} s_j \;\le\; |V|-K+i.
\]
Under the ordering, the configuration that maximizes $s_i$ given
$\sum_{j=1}^i s_j$ is to set $s_1=\cdots=s_i$, hence
\[
i\cdot s_i \;\le\; |V|-K+i
\quad\Longrightarrow\quad
s_i \;\le\; \frac{|V|-K+i}{i}.
\]
Integrality of $s_i$ yields
\[
s_i \;\le\; \Big\lfloor \frac{|V|-K+i}{i} \Big\rfloor,
\]
which matches the proposed inequality.
 
\paragraph{Tightness and Special Cases.}
The bound is tight, e.g., when $s_1=\cdots=s_i=\big\lfloor \frac{|V|-K+i}{i} \big\rfloor$
and $s_{i+1}=\cdots=s_K=1$ (whenever the sum equals $|V|$). For $i=1$ it reduces to
$s_1 \le \lfloor |V|-K+1 \rfloor$, and for $i=K$ it gives
$s_K \le \lfloor |V|/K \rfloor$, consistent with the interpretation that the smallest
block cannot exceed the average size.
% -----------------------------------------------------------------

\section{Valid Inequality 3}
\begin{equation}
    d_{vi} \geq \sum_{u \in CN(v)} x_{u,i} - |CN(v)| + 1 \quad \forall v \in V,\, \forall i \in \Pi
\end{equation}
\paragraph{Explanation} This inequality links the domination variable $d_{v,i}$ with the assignment of $v$’s neighbors to block $i$. Its logic is as follows:
 
\begin{itemize}

    \item If \textbf{all neighbors of $v$} are assigned to block $i$, then
    \[
    \sum_{u \in CN(v)} x_{u,i} = |CN(v)|,
    \]
    and the right-hand side equals $1$. Hence, the constraint enforces $d_{v,i} \ge 1$, which forces $d_{v,i} = 1$. That is, vertex $v$ must dominate block $i$.
 
    \item If at least one neighbor of $v$ is not in block $i$, then

    \[
    \sum_{u \in CN(v)} x_{u,i} \le |CN(v)|-1,
    \]

    and the right-hand side is nonpositive. Since $d_{v,i} \ge 0$, the inequality becomes redundant and places no restriction.

\end{itemize}
 
\paragraph{} Thus, the inequality ensures that \textbf{whenever block $i$ consists entirely of neighbors of $v$}, the model must recognize this by setting $d_{v,i}=1$. In other words, it eliminates the possibility that a vertex is ``silently'' dominating a block without the variable $d_{v,i}$ reflecting it.
% -----------------------------------------------------------------

\section{Valid Inequality 4}
\begin{equation}
    d_{vi} \leq \sum_{u \in CN(v)} x_{ui} \quad \forall v \in V,\, \forall i \in \Pi
\end{equation}
\paragraph{Explanation} This inequality ensures that vertex $v$ can only be declared as a dominator of block $i$ if at least one of its neighbors is assigned to block $i$. In other words, domination is not allowed in a vacuum: the variable $d_{v,i}$ cannot take the value $1$ unless there is some vertex in block $i$ that $v$ is actually adjacent to.  
 
\begin{itemize}

    \item If no neighbor of $v$ lies in block $i$, then the right-hand side equals $0$, and the constraint enforces $d_{v,i}=0$.

    \item If one or more neighbors of $v$ belong to block $i$, then the inequality permits $d_{v,i}$ to be $1$, but does not require it.

\end{itemize}
 
Thus, this constraint acts as a \textbf{logical upper bound}: it rules out false positives where a vertex is labeled as a dominator of a block that contains none of its neighbors.
 
% -----------------------------------------------------------------

\section{Valid Inequality 5}
\begin{equation}
    \sum_{v \in S} d_{vi} \leq 1 \quad \forall S \in \mathcal{I},\; \forall i \in \Pi
\end{equation}
\paragraph{Explanation} This inequality leverages structural information from the squared graph $G^2$, in which two vertices are adjacent if their distance in the original graph $G$ is at most two. By construction, a maximal independent set in $G^2$ consists of vertices that are pairwise at distance at least three in $G$. Such vertices cannot jointly dominate the same block, because domination requires adjacency relationships that are incompatible with this distance condition.  
 
Therefore, the inequality enforces that \textbf{at most one vertex from any maximal independent set of $G^2$ can dominate a given block}.  
 
\begin{itemize}
    \item If two or more vertices from the same independent set $S$ were both chosen as dominators of the same block, then the block would need to simultaneously be contained within the neighborhoods of these vertices. But since their neighborhoods are too far apart in $G$, this is impossible.
    \item The inequality thus cuts off such infeasible configurations at the linear relaxation level, strengthening the formulation without excluding any valid dominator partition.
\end{itemize}
 
This family of inequalities integrates higher-order graph structure into the model and systematically restricts domination assignments that are combinatorially impossible.
% -----------------------------------------------------------------

\section{Valid Inequality 6}
\begin{equation}
    \sum_{u \in CN(v)} x_{ui} \leq |CN(v)| - 1 + d_{vi} - \sum_{\substack{y \in S \\ y \neq v}} d_{yi} \quad \forall v \in V,\; \forall S \in \mathcal{I} \text{ with } v \in S,\; \forall i \in \Pi
\end{equation}
\paragraph{Explanation} This inequality couples the ``all--neighbors--in--the--block'' trigger for $v$ with the exclusivity requirement induced by the maximal independent set $\mathrm{MIS}$ in $G^2$. Specifically:
 
\begin{itemize}
    \item If \textbf{all neighbors of $v$} lie in block $i$, then $\sum_{u \in CN(v)} x_{u,i} = |CN(v)|$, and the inequality becomes
    \[
    |CN(v)| \;\le\; |CN(v)| - 1 + d_{v,i} - \sum_{y \in \mathrm{MIS}\setminus\{v\}} d_{y,i},
    \]
    which simplifies to
    \[
    1 \;\le\; d_{v,i} - \sum_{y \in \mathrm{MIS}\setminus\{v\}} d_{y,i}.
    \]
    Hence, $d_{v,i}$ must be $1$ and \textbf{all other} $d_{y,i}$ with $y \in \mathrm{MIS}\setminus\{v\}$ must be $0$. In words: when block $i$ is entirely within $N(v)$, vertex $v$ is forced to be the \textbf{unique} dominator from $\mathrm{MIS}$ for that block.
 
    \item If \textbf{not all neighbors of $v$} are in block $i$, then $\sum_{u \in CN(v)} x_{u,i} \le |CN(v)|-1$, so the left-hand side is at most $|CN(v)|-1$, and the inequality becomes nonbinding: it does not force any particular value of $d_{v,i}$ (nor of the other $d_{y,i}$).
\end{itemize}
 
Thus, the constraint acts as a \textbf{cover-type coupling inequality}: it activates only in the event that block $i$ is fully contained in $N(v)$, and in that case it both (i) forces $v$ to dominate block $i$ and (ii) enforces \textbf{exclusivity} among the dominators drawn from the same maximal independent set of $G^2$.

% -----------------------------------------------------------------

\section{Valid Inequality 7}

\paragraph{Block size bounded by dominators' degrees (UnionCap).}
For all $i\in\Pi$,
\[
\sum_{u\in V} x_{u,i}
\;\le\;
|V|\Bigl(1-\sum_{v\in V} d_{v,i}\Bigr)
\;+\;
\sum_{v\in V} \bigl(1+|N(v)|\bigr)\, d_{v,i}.
\]
