\section{Valid Inequality 1}
\begin{equation}
    \sum_{v \in V} x_{vi} \geq \left\lceil \frac{|V|}{k} \right\rceil
\end{equation}
\paragraph{Explanation} This constraint requires the first block of the partition to contain at least 
$\lceil |V|/K \rceil$ vertices. Since the vertex set $V$ is divided into exactly 
$K$ nonempty blocks, the average block size is $|V|/K$. By the pigeonhole principle, 
at least one block must contain at least $\lceil |V|/K \rceil$ vertices. Assigning 
this requirement to block $1$ is valid because of the symmetry-breaking constraints 
that enforce a nonincreasing block size order:
\[
\sum_{v \in V} x_{v,i} \;\geq\; \sum_{v \in V} x_{v,i+1}, \quad 
\forall i \in \{1,\dots,K-1\}.
\]
 
Thus, block $1$ can be designated as the largest block without loss of generality. 
This inequality preserves feasibility, strengthens the LP relaxation by ruling out 
artificially balanced fractional solutions, and reduces solver symmetry by eliminating 
weakly balanced partitions.

\section{Valid Inequality 2}
\begin{equation}
    \sum_{v \in V} x_{vi} \leq \left\lfloor \frac{|V| - K + i}{i} \right\rfloor \quad \forall i \in \Pi
\end{equation}
\paragraph{Explanation} Let $s_j := \sum_{v\in V} x_{v,j}$ denote the size of block $j$. The model enforces
(i) nonemptiness $s_j \ge 1$ for all $j$, (ii) a nonincreasing order
$s_1 \ge s_2 \ge \cdots \ge s_K$, and (iii) $\sum_{j=1}^K s_j = |V|$.
Fix $i$. Since the last $K-i$ blocks must contain at least $K-i$ vertices in total,
the first $i$ blocks can use at most $|V|-(K-i)$ vertices:
\[
\sum_{j=1}^{i} s_j \;\le\; |V|-K+i.
\]
Under the ordering, the configuration that maximizes $s_i$ given
$\sum_{j=1}^i s_j$ is to set $s_1=\cdots=s_i$, hence
\[
i\cdot s_i \;\le\; |V|-K+i
\quad\Longrightarrow\quad
s_i \;\le\; \frac{|V|-K+i}{i}.
\]
Integrality of $s_i$ yields
\[
s_i \;\le\; \Big\lfloor \frac{|V|-K+i}{i} \Big\rfloor,
\]
which matches the proposed inequality.
 
\paragraph{Tightness and Special Cases.}
The bound is tight, e.g., when $s_1=\cdots=s_i=\big\lfloor \frac{|V|-K+i}{i} \big\rfloor$
and $s_{i+1}=\cdots=s_K=1$ (whenever the sum equals $|V|$). For $i=1$ it reduces to
$s_1 \le \lfloor |V|-K+1 \rfloor$, and for $i=K$ it gives
$s_K \le \lfloor |V|/K \rfloor$, consistent with the interpretation that the smallest
block cannot exceed the average size.

\section{Valid Inequality 3}
\begin{equation}
    d_{vi} \geq \sum_{u \in CN(v)} x_{u,i} - |CN(v)| + 1 \quad \forall v \in V,\, \forall i \in \Pi
\end{equation}
\paragraph{Explanation}

\section{Valid Inequality 4}
\begin{equation}
    d_{vi} \leq \sum_{u \in CN(v)} x_{ui} \quad \forall v \in V,\, \forall i \in \Pi
\end{equation}
\paragraph{Explanation}

\section{Valid Inequality 5}
\begin{equation}
    \sum_{v \in S} d_{vi} \leq 1 \quad \forall S \in \mathcal{I},\; \forall i \in \Pi
\end{equation}
\paragraph{Explanation}

\section{Valid Inequality 6}
\begin{equation}
    \sum_{u \in CN(v)} x_{ui} \leq |CN(v)| - 1 + d_{vi} - \sum_{\substack{y \in S \\ y \neq v}} d_{yi} \quad \forall v \in V,\; \forall S \in \mathcal{I} \text{ with } v \in S,\; \forall i \in \Pi
\end{equation}
\paragraph{Explanation}
