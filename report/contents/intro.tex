\section{Introduction}
\label{sec:introduction}


\paragraph{} Graph partitioning problems arise in various applications, including social network analysis and data clustering. One such problem is the \textit{Dominator Partition Problem} (DPP), which involves partitioning the vertex set of a graph into $k$ disjoint partitions such that every vertex dominates at least one partition.

\paragraph{} The goal of this work is twofold: first, to provide a formal integer programming formulation of the DPP, and second, to develop and analyze valid inequalities that improve solver performance. Computational experiments are conducted to evaluate the impact of these inequalities on solution time.
