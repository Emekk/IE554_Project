\section{Impact of Valid Inequalities to Solver Performance}
\label{sec:impact_of_valid_inequalities}


\paragraph{} To assess the computational effectiveness of the valid inequalities introduced in the previous section, we conducted a series of experiments on large graph instances.

\paragraph{} We generated random undirected graphs with $n = 100 \text{ and } 200$ nodes using an Erdős–Rényi model, where each edge is independently included with probability $p = 0.2$. We also applied a connectivity post-process to retain only connected graphs.

\paragraph{} For the generated graphs, the Dominator Partition Problem is solved for $k = 10$ for $n=100$ and $k = 50$ for $n=200$ under three different configurations:
\begin{itemize}
    \item[(i)] No valid inequalities
    \item[(ii)] With valid inequalities \textsl{\eqref{eq:valid1}} and \textsl{\eqref{eq:valid2}} only
    \item[(iii)] With all valid inequalities
\end{itemize}

\paragraph{} The solution times (in seconds) for each case are reported in below tables.
\begin{table}[H]
\centering
\caption{Impact of Valid Inequalities on Solver Performance ($n=100$, $p=0.2$, $k=10$)}
\label{tab:performance1}
\begin{tabular}{l|c|c}
\textbf{Model Variant} & \textbf{Time (s)} & \textbf{Improvement (\%)} \\
\hline
No valid inequalities        & TODO & -- \\
Valid inequalities 1 \& 2    & 680.52 & TODO\% \\
All valid inequalities       & 651.63 & TODO\% \\
\end{tabular}
\end{table}

\begin{table}[H]
\centering
\caption{Impact of Valid Inequalities on Solver Performance ($n=200$, $p=0.2$, $k=50$)}
\label{tab:performance2}
\begin{tabular}{l|c|c}
\textbf{Model Variant} & \textbf{Time (s)} & \textbf{Improvement (\%)} \\
\hline
No valid inequalities        & 1047.77 & -- \\
Valid inequalities 1 \& 2    & 680.52 & 35.05\% \\
All valid inequalities       & 651.63 & 37.81\% \\
\end{tabular}
\end{table}

\paragraph{} As seen in the results, the addition of valid inequalities significantly enhances solver performance. Incorporating just the first two inequalities reduces runtime by more than one-third. When all proposed inequalities are applied, the model achieves a total runtime reduction of up to TODO\%, demonstrating the power of carefully crafted inequalities in accelerating the solution process on large-scale instances.
