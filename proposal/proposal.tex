\documentclass[12pt]{article}
\usepackage[utf8]{inputenc}
\usepackage{enumitem}
\usepackage{graphicx}
\usepackage[autostyle]{csquotes}
\usepackage{amsmath}
\usepackage[T1]{fontenc}
\usepackage{romannum}
\usepackage{calligra} % handwritten font
%\usepackage{newtxtext} % For text font
\usepackage{newtxmath}
\usepackage{float}
%\usepackage{pdfpages}
\usepackage{pifont}
\usepackage{setspace}
\usepackage{subcaption}
\usepackage[export]{adjustbox}
\usepackage{tcolorbox}
\usepackage{multicol}
%\usepackage{pgfplots}  % matplotlib plots
%\usepackage{tikz}
%\usetikzlibrary{shapes.geometric, shapes.multipart, arrows, positioning, arrows.meta}
\usepackage[bottom]{footmisc}  % fix footnotes at the bottom of the page
\usepackage{hyperref}
\usepackage[a4paper, margin=1cm]{geometry}
%\usepackage[toc, section=section]{glossaries}
\usepackage[all]{hypcap}  % needed to help hyperlinks direct correctly

% hyperlinks
\hypersetup{
    colorlinks=true, %set true if you want colored links
    linktoc=all,     %set to all if you want both sections and subsections linked
    linkcolor=black, %choose some color if you want links to stand out
    citecolor=black,
}

% table of contents
\renewcommand\contentsname{CONTENTS}

\setstretch{1.1}
\setlength{\parskip}{0pt}
\setlength{\parindent}{0pt}

\begin{document}
\pagenumbering{gobble}

\begin{center}
    \Large \textbf{IE 554 Project Proposal} \\
    \normalsize \textsl{Emek Irmak \& Ömer Turan Şahinaslan}
\end{center}

\textbf{\large Problem Description}
\vspace*{-2mm}

\paragraph{} Graph partitioning is a key topic in optimization. We study the \textbf{Dominator Partition Problem}, introduced by Hedetniemi and Haynes (2006) \cite{dominator_partitions}, and propose an exact integer programming model with potential improvements.

\paragraph{} In graph $G = (V, E)$, a vertex $v$ \textit{dominates} set $S \subseteq V$ if it's adjacent to all $u \in S$. A \textbf{dominator partition} divides $V$ into $k$ blocks, such that each $v \in V$ dominates at least one of the blocks. The goal is to find the smallest such $k$, called $\pi_d(G)$.

\vspace{5mm}

\textbf{\large Our Plan}
\vspace*{-2mm}

\paragraph{} We gave the following IP model for the dominator partition problem for a fixed $k$:

\hspace{4mm}
\scalebox{0.95}{%
  \parbox{\linewidth}{%
    \vspace{4mm}
\begin{minipage}[t]{0.48\textwidth}
\textbf{Sets \& Parameters}
\begin{itemize}[label=, noitemsep, topsep=0pt, leftmargin=2mm]
    \item $V$: set of vertices in the graph, of size $n$
    \item $a_{vu}$: 1 if vertices $v$ and $u$ are adjacent
    \item $k$: number of blocks in the partition
\end{itemize}
\end{minipage}
\hfill
\begin{minipage}[t]{0.48\textwidth}
\textbf{Decision Variables}
\begin{itemize}[label=, noitemsep, topsep=0pt, leftmargin=2mm]
    \item $x_{vi}$: 1 if vertex $v$ is assigned to the $i$\textsuperscript{th} block
    \item $d_{vi}$: 1 if vertex $v$ dominates block $i$
\end{itemize}
\end{minipage}

\subsection*{Objective Function \& Constraints}
\begin{align*}
    \min \quad &0 &&\text{(no objective)}\\
    \text{s.t.} \quad
    &\sum_{i=1}^{k} x_{vi} = 1, \quad \forall v \in V &&\text{(each vertex is assigned to one block)}\\
    &\sum_{v \in V} x_{vi} \geq 1 \quad \forall i \in \{1, 2, \dots, k\} &&\text{(no empty blocks)}\\
    &x_{ui} \leq a_{vu} + (1 - d_{vi}) \quad \forall u,v \in V, i \in \{1, 2, \dots, k\} &&\text{(domination condition)}\\
    &\sum_{i=1}^{k} d_{vi} \geq 1 \quad \forall v \in V &&\text{(each vertex dominates at least one block)}\\
    & \sum_{v \in V} x_{vi} \geq \sum_{v \in V} x_{v,i+1} \quad \forall i \in \{1, 2, \dots, k-1\} &&\text{(blocks are used in order)}\\
    &x_{vi}, d_{vi} \quad \forall v \in V, i \in \{1, 2, \dots, k\} &&\text{(binary variables)}\\
\end{align*}
%
  }%
}

\vspace{-6mm}
\paragraph{} The planned contributions are:
\begin{itemize}[label=, itemsep=0pt]
    \item \textbf{Exact IP Model}: First IP formulation for dominator partitioning.
    \item \textbf{Model Strengthening}: Add valid inequalities.
    \item \textbf{Experiments}: Compare original vs. improved models on graphs of various sizes.
\end{itemize}

\textbf{\large Literature Review}
\vspace*{-2mm}

\paragraph{} Most existing work is theoretical, focused on complexity and special graph classes \cite{dominator_partitions}. Determining $\pi_d(G)$ is NP-complete, but no exact IP formulation exists in prior literature. Our study fills this gap.

\small
\bibliographystyle{plain}
\begin{thebibliography}{9}
\bibitem{dominator_partitions}
S. M. Hedetniemi et al., \textit{Dominator Partitions of Graphs}, 2008.
\end{thebibliography}

\end{document}